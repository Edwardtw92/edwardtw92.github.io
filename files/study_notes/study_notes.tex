\documentclass[11pt]{article}
\input{~/Documents/Undergrad/preamble.tex}
\usepackage[left=1.5cm, right=1.5cm, top=1.3cm, includehead, includefoot]{geometry}

%% header
\pagestyle{fancy}
\fancyhead[L]{\bf\large PHYS 476: Introduction to General Relativity}
\fancyhead[R]{\bf\large Edward \(\mapsto\) Michelle}
% \fancyfoot[C]{Page \thepage\ of 2}
\renewcommand{\headrulewidth}{0.4pt}
\renewcommand{\footrulewidth}{0.4pt}
\setlength{\headheight}{35pt}
%==============================================================================================

\begin{document}
\begin{titlepage}
	\begin{center}
		\vspace*{4cm}
		\textbf{AMATH 475}

		University of Waterloo

		Taught by: Eduardo Martin-Martinez

		Updated: \today

		\vspace{1.5cm}
		\textbf{Edward Chang}
	\end{center}
\end{titlepage}
\tableofcontents
\newpage
%==============================================================================================

\fancyhead[L]{\bf\large PHYS 476: Introduction to General Relativity\\ \small{\leftmark}}
\fancyhead[R]{\bf\large Edward Chang (For Michelle)\\ \small{\rightmark}}
\renewcommand{\sectionmark}[1]{\markboth{#1}{}}
\renewcommand{\subsectionmark}[1]{\markright{#1}}
% \fancyfoot[C]{Page \thepage\ of 2}
\renewcommand{\headrulewidth}{0.4pt}
\renewcommand{\footrulewidth}{0.4pt}
\setlength{\headheight}{35pt}
%==============================================================================================

\section{Index Notation}
\begin{defm}{Covariant}
	Objects that transforms under change of basis like the element of the basis are called \textbf{covariant}, and its components have sub-indicies.
\end{defm}

\begin{defm}{Contravariant}
	Objects that transforms under change of basis like components of vectors are called \textbf{contravariant} and its components have super-indicies.
\end{defm}

\begin{defm}{Dual Basis \& Dual Space}
	The \textbf{dual basis} to \(\mathcal{B} = \left\{ \mathsf{e}_{1}, \dots, \mathsf{e} _{n} \right\} \) is \(\mathbb{B} ^{\star} = \left\{ \mathsf{e}^{1}, \dots, \mathsf{e}^{n} \right\}\) and it is the collection of linear operators such that \(\mathsf{e}^{i} (\mathsf{e}_{j}) = \delta ^{i} _{j}\). \(\mathbb{B}^{\star}\) spans a vector space \(V ^{\star}\) called the \textbf{dual space} of \(V\).
\end{defm}
\newpage
%==============================================================================================

\section{Special Relativity}
\subsection{Postulates of Special Relativity}
\begin{defn}{Event}
	An \textbf{event} is an individual point in spacetime, usually labelled which we represent by the tuple \(E \equiv \left( t, \vec{x} \right) \), \(t\) is the \textit{time coordinate} and \(\vec{x} \equiv \left( x, y, z \right) \) is the \textit{space coordinates}.
\end{defn}

\begin{defn}{Spacetime}
	\textbf{Spacetime} is the set of all events, \(\mathbb{S} = \left\{ E \equiv \left( t, \vec{x} \right): t \in R, \vec{x} \in \R ^{3}  \right\} \).
\end{defn}
\begin{defm}{Reference Frame}
	A \textbf{reference frame}, establishes a spacetime coordinate system \(\left(t, \vec{x} \right) \), which is a spatial coordinate system where the position of point-like particles can be specified, and a clock (something that can measure time).
\end{defm}

\begin{defm}{Inertial Reference Frame (IRF)}
	An \textbf{inertial reference frame} is a reference frame for which a particle stationary at it origin experience no force (Newton's first law holds).
\end{defm}

\begin{defm}{Postulates of Special Relativity 1 (SR): Principle of Relativity}
	\begin{enumerate}[leftmargin=*, label=\arabic*.]
		\item \textbf{Principle of Relativity:} In the absence of gravity, all the laws of physics are identical in all inertial reference frames (This postulate is also in Galilean relativity).
		\item \textbf{Speed of Light is Constant and Equal:} The speed of light in vacuum ``\(c\)'' is constant and the same for all inertial reference frames (absense of gravity).
	\end{enumerate}
\end{defm}

\subsection{Lorentz Transformation}
\begin{defn}{Galilean Transformation}
	Consider two inertial frames \(S \equiv \left( t, \vec{x} \right) \) and \(S' \equiv \left( t', \vec{x}' \right) \), where \(S'\) moves with velocity \(\vec{v} = v \hat{x}\) to the right relative to \(S\).
	The \textbf{Galilean transformation} (Galilean boost) from \(S\) to \(S'\) is:
	\[
		\begin{cases}
			x' = x - v t\\
			y' = y\\
			z' = z\\
			t' = t
		\end{cases}
	\]
	Note: Galilean transformation would result in mathematical inconsistency (todo).
\end{defn}

\begin{defm}{Spacetime Interval}
	Given a particular inertial reference frame that establishes space time coordinates \(\left( t, \vec{x} \right) \), the \textbf{spacetime interval} between two events \(E _{1} \equiv (t _{1}, \vec{x} _{1})\) and \(E _{2} \equiv \left( t _{2}, \vec{x} _{2} \right) \) is \(\Delta s ^{2} := - c ^{2} \Delta t ^{2} + \Delta \vec{x} ^{2} = - c ^{2} \left( t _{2} - t _{1} \right) ^{2} + \left( \vec{x} _{2} - \vec{x} _{1} \right) ^{2} \).
	\begin{noot}
		In this course we use the signature \(\left( -, +, +, + \right) \).
	\end{noot}
	\begin{remarkbox}
		If the two events are connected by the propagation of light, \(\Delta \vec{x} ^{2} = c ^{2} \Delta t ^{2} \implies \Delta s ^{2} = - c ^{2} \Delta t ^{2} + c ^{2} \Delta t ^{2} = 0\).
	\end{remarkbox}
\end{defm}

\begin{noot}
	Possible transformations between two inertial reference frames: 3 rotations, 3 translations, 1 times shift, 3 boosts.

	However, the spatial distance \(\| \vec{x} _{2} - \vec{x} \|\) is invariant under rotations and translations. Similarly, \(\Delta t = t _{2} - t _{1}\) is invariant under time shift. So the only transformations with significance are the boost.
\end{noot}
\newpage
%==============================================================================================

\section{Differential Geometry}
\subsection{Topology}
\begin{defn}{Topology}
	A \textbf{topology on a set \(X\)}, \(S\) is a collection of open sets of \(X\), and \(S\) is a subset of the power set of \(X\).
	\(S\) satisfies:
	\begin{enumerate}[label=\arabic*)]
		\item \(\varnothing, X \in S\).
		\item Any union of elements of \(S\) is in \(S\).
		\item Any finite intersection of  elements of \(S\) is in \(S\).
	\end{enumerate}
\end{defn}

\begin{defm}{Topological Space}
	A \textbf{topological space} \(\left( X, S \right) \), is an ordered pair where \(X\) is a set and \(S\) is a topology of \(X\).
\end{defm}

\subsection{Manifolds}
\begin{defn}{Homeomorphism}
	A \textbf{homeomorphism} between two topological spaces \(X\) and \(Y\) is a map \(\sigma: X \to Y\) such that the map is topologically continuous and it's inverse is topologically continuous (\(\phi\) is a bijection).
\end{defn}

\begin{defn}{Hausdorff Space}
	A topological space \(X\) is a \textbf{Hausdorff space} if for all distinct points \(x, y \in X\) there exists neighbourhoods of \(x\) and \(y\), \(\mathcal{U} _{x}, \mathcal{U} _{y}\) respectively, such that \(\mathcal{U} _{x} \cap \mathcal{U} _{y} = \varnothing\).
\end{defn}

\begin{defm}{Topological Manifolds}
	A \textbf{topological manifold} of dimension \(n\), \(\mathcal{M}\) is a topological space that is \textbf{Hausdorff} and at every point possess an open neighbourhood homeomorphic to \(\R ^{n}\).

	Note: The proper definition also requires that the space is second countable.
\end{defm}

\begin{defn}{Charts}
	A \textbf{chart} \(\left( \mathcal{U} _{\alpha}, \phi _{\alpha} \right) \), where \(\phi _{\alpha}\) is a homeomorphism from an open subset \(\mathcal{U} _{\alpha} \subseteq \mathcal{M}\) to \(\R ^{n}\). Note: \(\phi _{\alpha}: \mathcal{U} \underset{open}{\subseteq} M \to \R ^{n}\), and \(\phi _{\alpha} (\mathsf{x}) \equiv x ^{\mu}\), \(x ^{\mu}\) is the coordinate in \(\R ^{n}\).
\end{defn}

\begin{defn}{Transition Map}
	Consider two charts of \(\mathcal{M}\), \(\left( \mathcal{U} _{\alpha}, \phi _{\alpha} \right) \) and \(\left( \mathcal{U} _{\beta}, \phi _{\beta} \right) \), where \(\mathcal{U} _{\alpha} \cap \mathcal{U} _{\beta} \neq \varnothing\). The \textbf{transition map} is a homeomorphism from \(\phi _{\alpha} \left( \mathcal{U} _{\alpha} \cap \mathcal{U} _{\beta} \right) \to \phi _{\beta} \left( \mathcal{U} _{\alpha} \cap \mathcal{U} _{\beta} \right)  \), defined by \(\phi _{\beta} \circ \phi _{\alpha} ^{-1}\).
\end{defn}

\begin{defn}{Smooth Atlas}
	A \textbf{smooth atlas} is a set of charts covering the whole manifold such that the transition map \(\phi _{\alpha} \circ \phi _{\beta} ^{-1}\) are \(C ^{\infty}\).
\end{defn}

\begin{defm}{Smooth Manifold}
	A \textbf{smooth manifold} is a topological manifold that has a smooth atlas.
\end{defm}

\subsection{Structures on Manifold}
\subsubsection{Curves and Functions}
\begin{defm}{Curve on Manifold}
	A \textbf{curve} on a manifold \(\mathcal{M}\) is a smooth and invertible map \(\gamma : \R \to M\).

	Note: check if it is actually smooth and invertible.
\end{defm}

\begin{defm}{Functions on Manifold}
	A \textbf{function} on a manifold \(\mathcal{M}\) is a map \(f: M \to \R\).

	\(\bar{f} := f \circ \phi ^{-1}: \R ^{n} \to \R\).

	Note that \(f\) and \(\bar{f}\) are not necessarily invertible, \(f\) is smooth if and only if \(\bar{f}\) is \(C ^{\infty}\).
\end{defm}

\subsubsection{Vectors}
\textbf{Motivation:} We want to be able to define the notion of direction on manifold \(\mathcal{M}\). To do so we want to relate derivatives and curves to directions.

\begin{defm}{Coordinate Curves}
	\textbf{Coordinate curves} of a chart are the image of the axes in \(\R ^{n}\) under \(\phi ^{-1}\).
\end{defm}

\begin{defm}{Tangent Vector}
	The \textbf{tangent vector} to \(\gamma\) at \(\tau _{0}\) is \(\partial _{\tau} | _{\tau _{0}} = \frac{d}{d \tau} | _{\tau _{0}} = \frac{d x ^{\mu}}{d \tau} \partial _{\mu} | _{\tau _{0}}\).

	And \(\Upsilon _{\mu} (f) | _{\tau _{0}} = \partial _{\mu} \bar{f} | _{\tau _{0}}\).

\end{defm}

\begin{defm}{Tangent Space \(T _{p} \mathcal{M}\)}
	The set \(T _{p} \mathcal{M}\) is the \textbf{tangent space} of all the vectors \(\mathsf{v}_{p}\) at a point \(p \in M\), and it has dimension equal to \(\dim(\mathcal{M})\). We call \(T _{p} \mathcal{M}\) the \textbf{tangent space to \(\mathcal{M}\) at \(p \in \mathcal{M}\)}.
\end{defm}

\begin{defn}{Coordinate Basis}
	The set \(\left\{ \Upsilon _{p\, \mu} \right\} \) is the \textbf{coordinate basis} of \(T _{p} \mathcal{M}\) is the set of vectors tangent to the coordinate curves of chart \(\left( \mathcal{U}, \phi \right) \).
	\begin{noot}
		Every chart comes with a coordinate basis and every coordinate basis defines a chart.
	\end{noot}
\end{defn}

\begin{thm}{Coordinate Basis and Commutativity}	
	A basis of \(\left\{ \Upsilon _{\mu} \right\} \) \(T _{p} ^{\star} \mathcal{M}\) is a coordinate basis \(\iff \left[ \Upsilon _{\mu}, \Upsilon _{\nu} \right] = 0\).
\end{thm}

\begin{defm}{Basis Transformation}
	Consider \(\frac{\partial}{\partial x ^{\prime \mu}} = \frac{\partial x ^{\nu}}{\partial x ^{\prime \mu}} \frac{\partial}{\partial x ^{\nu}}\). Implying \(\partial ' _{\mu} = \Lambda \indices{_{\mu} ^{\nu}} \partial _{\nu}\) with \(\Lambda \indices{_{\mu} ^{\nu}} = \frac{\partial x ^{\nu}}{\partial x ^{\prime \mu}}\).
\end{defm}

\begin{defm}{Vector Field}
	A \textbf{vector field} over \(\mathcal{M}\) is a set of vectors of \(T _{p} \mathcal{M}\) for each \(p \in \mathcal{M}\) such that their components in any coordinate basis are smooth functions. Note that vector fields follows Leibniz rule.
\end{defm}

\begin{defm}{Composition of Vector Fields}
	\(\left( \mathsf{v} \circ \mathsf{w} \right) (f) := \mathsf{v} \left[ \mathsf{w} (f) \right]  \). The composition of vector fields does not obey the Leibniz rule and hence is not a vector field.
	\begin{noot}
		\(\left( \mathsf{v} \circ \mathsf{w} \right) \left( f g \right) = f \cdot \left( \mathsf{v} \circ \mathsf{w} \right) (g) + g \cdot \left( \mathsf{v} \circ \mathsf{w} \right)  (f) + \mathsf{w} (f) \cdot \mathsf{v} (g) + \mathsf{v} (f) \cdot \mathsf{w} (g) \neq f \cdot \left( \mathsf{v} \circ \mathsf{w} \right)(g) + g \cdot \left( \mathsf{v} \circ \mathsf{w} \right) (f)\).
	\end{noot}
\end{defm}

\begin{defm}{Lie Bracket of Vector Fields}
	The \textbf{Lie brackets of vector field} is a binary operator such that \(\left[ \cdot, \cdot  \right] : (A, B) \mapsto AB - BA\). And Lie brackets satisfies:
	\begin{enumerate}[label=\arabic*)]
		\item Antisymmetry: \(\left[ \mathsf{v}, \mathsf{w} \right] = - \left[ \mathsf{w}, \mathsf{v} \right] \).
		\item Jacobi Identity: \(\left[ \mathsf{u}, \left[ \mathsf{v}, \mathsf{w} \right]  \right] + \left[ \mathsf{w}, \left[ \mathsf{u}, \mathsf{v} \right]  \right] + \left[ \mathsf{v}, \left[ \mathsf{w}, \mathsf{u} \right]  \right] = 0\)
	\end{enumerate}
	The Lie bracket of vector fields is a vector field.
	\begin{remarkbox}
		In a coordinate basis, \(\left\{ \Upsilon _{\mu} \right\} = \left\{ \partial _{\mu} \right\} \), \(\left[ \mathsf{v}, \mathsf{w} \right] ^{\mu} = v ^{\nu} \partial _{\nu} w ^{\mu} - w ^{\nu} \partial _{\nu} v ^{\mu}\).
	\end{remarkbox}
\end{defm}

\subsubsection{1-Forms}
\begin{defm}{1-Form}
	A \textbf{1-form}, \(\mathsf{\omega}\), is a real linear functional over \(T _{p} \mathcal{M}\), \(\mathsf{\omega}: T _{p} \mathcal{M} \to \R\) defined by \(\mathsf{\omega : T _{p} \mathcal{M} \to \R}\) defined by \(\mathsf{\omega}: \mathsf{v} \mapsto \left< \mathsf{\omega}, \mathsf{v} \right>\). Note: 1-forms are sometimes called covariant vectors.
\end{defm}

\begin{defn}{Cotangent Space and Dual Basis}
	Given an arbitrary basis \(\left\{ \mathsf{e} _{a} \right\} \) of \(T _{p} \mathcal{M}\). There exists an unique set of 1-forms \(\left\{ \mathsf{e} ^{a} \right\} \) such that \(\left<\mathsf{e} ^{a}, \mathsf{e} _{b} \right> = \delta _{b} ^{a}\). This set is linear independent and forms a basis of the \textbf{cotangent space} \(T _{p} ^{\star} \mathcal{M}\). We call \(\left\{ \mathsf{e} _{a} \right\} \) the \textbf{dual basis}.
	\begin{noot}
		Elements of \(T _{p} \mathcal{M}\) also acts linearly on \(T _{p} ^{\star} \mathcal{M}\) \(\implies  T _{p} ^{\star \star} \mathcal{M} = T _{p} \mathcal{M}\).
	\end{noot}
	\begin{remarkbox}
		\(\left< \mathsf{\omega}, \mathsf{v} \right> = \left< \omega _{a} \mathsf{e ^{a}}, v ^{b} \mathsf{e} _{b} \right> = \omega _{a} v ^{b}\).

		\(\mathsf{e}' _{a} = \Lambda \indices{_{a} ^{b}} \mathsf{e} _{b}\) and \(\mathsf{e} ^{ \prime a} = \tilde{\Lambda} \indices{^{a} _{b}} \mathsf{e} ^{b}\).
	\end{remarkbox}
\end{defn}

\begin{defm}{Differential}
	Each functions \(f\) over \(\mathcal{M}\) defines a 1-form \(\mathsf{d} f | _{p}\) at the point \(p \in \mathcal{M}\) that we call the \textbf{differential} of \(f\). \(\mathsf{d} f\) is defined by \(\left<\mathsf{d} f, \mathsf{v}\right> = \mathsf{v}(f)\). Note that \(f: \mathcal{M} \to \R\).

	In a coordinate basis \(\left\{ \Upsilon _{\mu} \right\} \), \(df = \partial _{\mu} \bar{f} \Upsilon ^{\mu} = \partial _{\mu} f \Upsilon ^{\mu}\).
\end{defm}

\begin{defm}{Coordinate Basis (Dual)}
	The coordinate basis of \(T _{p} ^{\star} \mathcal{M}\), \(\left\{ \Upsilon ^{\mu} \right\} \) is often represented by \(\left\{ \mathsf{d} x ^{\mu} \right\} \). (Can see by considering \(f(\mathsf{x}) = x ^{\mu} (\mathsf{x})\), note that this \(\mu\) is not being summed over.)

	\begin{remarkbox}
		\(\Upsilon ^{ \prime \mu} = \tilde{\Lambda} \indices{^{\mu} _{\nu}} \Upsilon ^{\nu} \iff \mathsf{d} x ^{ \prime \mu} = \tilde{\Lambda} \indices{^{\mu} _{\nu}} d x ^{\nu}\).
		
		\(\left< \Upsilon ^{ \prime \mu}, \Upsilon ' _{\nu} \right> = \left< \Upsilon ^{\mu}, \Upsilon _{\nu} \right> = \delta _{\nu} ^{\mu}\)

		\(\omega ' _{\mu} = \Lambda \indices{_{\mu} ^{\nu}} \omega _{\nu}\)
	\end{remarkbox}
\end{defm}

\subsubsection{Tensors}
\begin{defm}{Tensor of Type \(\left( r, s \right) \)}
	A \textbf{tensor of type \(\left( r, s \right) \)} is a multilinear map that acts on the vector space \(\left( T _{p} \right) ^{r} _{s} \mathcal{M} = \left( T _{p} ^{\star} \mathcal{M} \right) ^{\times r} \times \left( T _{p} \mathcal{M} \right) ^{\times s}\). Tensor of type \(\left( r, s \right) \) are called \(r\)-times contravariant and \(s\)-times covariant.
\end{defm}

\begin{defm}{Components of Tensor and Action of Tensor}
	A tensor \(\mathsf{T} \in \left( T _{p} \right) ^{r} _{s} \mathcal{M}\) is completely characterized by its action on a basis of \(\left( T _{p} \right) ^{s} _{r} \mathcal{M}\).

	The \textbf{components} of \(\mathsf{T}\) are \(\mathsf{T} \left( \Upsilon ^{\alpha _{1}}, \dots, \Upsilon ^{\alpha _{r}}, \Upsilon _{\beta _{1}}, \dots, \Upsilon _{\beta _{s}} \right) = T \indices{ ^{\alpha _{1} \dots \alpha _{r}} _{\beta _{1} \dots \beta _{s}}}\).

	The \textbf{action} of \(\mathsf{T}\) on arbitrary 1-forms and vectors is \(\mathsf{T} \left( \mathsf{\omega}, \mathsf{\sigma}, \dots, \mathsf{v}, \mathsf{w}, \dots \right) = T \indices{ ^{a b \dots} _{c d \dots}} \omega _{a} \sigma _{b} \dots v ^{c} w ^{d}\).
\end{defm}

\begin{defm}{Transformation of Tensor Components}
	Examples:
	\begin{enumerate}[label=\arabic*)]
		\item \({T'} \indices{^{a} _{b}} = \mathsf{T} \left( \tilde{\Lambda} \indices{^{a} _{c}} \Upsilon ^{c}, \Lambda \indices{_{b} ^{d}} \Upsilon _{d} \right) = \tilde{\Lambda} \indices{^{a} _{c}} \Lambda \indices{_{b} ^{d}} \mathsf{T} \left( \Upsilon ^{c}, \Upsilon _{d} \right) = \tilde{\Lambda} \indices{^{a} _{c}} \Lambda \indices{_{b} ^{d}} T \indices{^{c} _{d}}\).
		\item \({T'} \indices{_{c} ^{a b}} = \Lambda \indices{ _{c} ^{d}} \tilde{\Lambda} \indices{^{a} _{e}} \tilde{\Lambda} \indices{^{b} _{f}} T \indices{_{d} ^{e f}}\).
	\end{enumerate}
	Question: Ordering of indices?
\end{defm}

\subsubsection{Tensor Operations}
\begin{defn}{Symmetrization}
	\(T \indices{^{a _{1} \dots a _{r}} _{c _{1} \dots c _{t} \left( b _{1} \dots b _{s} \right) }} := \frac{1}{s !} \sum _{\pi} T \indices{^{a _{1} \dots a _{r}} _{c _{1} \dots c _{t} \pi(b _{1}) \dots \pi(b _{s})}}\).

	Examples:
	\begin{enumerate}[label=\arabic*)]
		\item \(T \indices{_{\alpha \left( \mu \nu \right) }} = \frac{1}{2 !} \left( T \indices{_{\alpha \mu \nu}} + T \indices{_{\alpha \nu \mu}} \right) \).
		\item \(T \indices{_{\mu (\nu}} R \indices{ _{\alpha ) \beta \gamma}} = \frac{1}{2 !} \left( T \indices{_{\mu \nu}} R \indices{_{\alpha \beta \gamma}} + T \indices{_{\mu \alpha}} R \indices{_{\nu \beta \gamma}} \right) \).
		\item \(T \indices{_{(\mu \alpha \beta)}} = \frac{1}{3 !} \left( T \indices{_{\mu \alpha \beta}} + T \indices{_{\mu \beta \alpha}} + T \indices{_{\alpha \mu \beta}} + T \indices{_{\alpha \beta \mu}} + T \indices{_{\beta \alpha \mu}} + T \indices{_{\beta \mu \alpha}}\right) \).
	\end{enumerate}
\end{defn}

\begin{defm}{Antisymmetrization}
	\(T \indices{^{a _{1} \dots a _{r}} _{c _{1} \dots c _{t} [b _{1} \dots b _{s}]}} := \frac{1}{s !} \sum _{\pi} \left( -1 \right) ^{\pi} \, T \indices{^{a _{1} \dots a _{r}} _{c _{1} \dots c _{t} \pi(b _{1}) \dots \pi (b _{s})}}\).

	Examples:
	\begin{enumerate}[label=\arabic*)]
		\item \(T \indices{_{[\mu \nu]}} = \frac{1}{2 !} \left( T \indices{_{\mu \nu}} - T \indices{_{\nu \mu}} \right) \).
		\item \(T \indices{_{[\mu \nu] \beta}} = \frac{1}{2 !} \left( T \indices{_{\mu \nu \beta}} - T \indices{_{\nu \mu \beta}} \right) \).
		\item \(T \indices{_{[\mu \alpha \beta]}} = \frac{1}{3 !} \left( T \indices{_{\mu \alpha \beta}} - T \indices{_{\mu \beta \alpha}} - T \indices{_{\alpha \mu \beta}} + T \indices{_{\alpha \beta \mu}} - T \indices{_{\beta \alpha \mu}} + T \indices{_{\beta \mu \alpha}}\right)\).
	\end{enumerate}
\end{defm}

\begin{defm}{Tensor Product}
	Let \(\mathsf{R} \in \left( T _{p} \right) ^{r} _{s} \mathcal{M}\), \(\mathsf{T} \in \left( T _{p} \right) ^{t} _{q} \mathcal{M}\). Then the \textbf{tensor product} of \(\mathsf{R}\) and \(\mathsf{T}\) is \(\mathsf{R} \otimes \mathsf{T} \in \left( T _{p} \right) ^{r + t} _{s + q} \mathcal{M}\) with components \(\left( \mathsf{R} \otimes \mathsf{T} \right) \indices{^{a _{1} \dots a _{r + t}} _{b _{1} \dots b _{s + q}}} := R \indices{^{a _{1} \dots a _{r}} _{b _{s + 1} \dots b _{s + q}}} \).
\end{defm}

\begin{defn}{Contraction}
	Let \(T \in \left( T _{p} \right) ^{r} _{s} \mathcal{M}\) with components \(T \indices{^{a _{1} \dots a _{r}} _{b _{1} \dots b _{s}}}\), we define the \textbf{contraction} of the first covariant and contravariant indices as the tensor \(\mathsf{R} \in \left( T _{p} \right) ^{r - 1} _{s - 1} \mathcal{M}\) as the tensor of components \(R \indices{^{a _{2} \dots a _{r}} _{b _{2} \dots b _{s}}} := T \indices{^{a a _{2} \dots a _{r}} _{a b _{1} \dots b _{s}}}\).
\end{defn}

\subsubsection{Integration on Manifold and Differential Forms}
\textbf{Motivation:} In a coordinate basis, \(\mathsf{\omega} = \omega _{\mu} \mathsf{d} x ^{\mu}\). Notice that a curve \(\gamma\) on the manifold \(\mathcal{M}\) is a submanifold of \(\mathcal{M}\). The integral of \(\mathsf{\omega}\) along the curve \(\gamma\) is \(\int _{\gamma \subset \mathcal{M}} \mathsf{\omega} = \int _{\gamma} \omega _{\mu} \, \mathsf{d} x ^{\mu} = \int_{s _{1}}^{s _{2}} \omega _{\mu} \frac{d x ^{\mu} (s)}{d s} \, ds\).

\begin{defn}{2-Form}
	A \textbf{2-form}, \(\mathsf{\sigma}\) is an antisymmetric twice covariant (type \(\left( 0, 2 \right) \)) tensor such that \(\mathsf{\sigma} \left( \mathsf{v}, \mathsf{w} \right) = - \mathsf{\sigma} \left( \mathsf{w}, \mathsf{v} \right)  \) and in a coordinate basis, \(\mathsf{\sigma} = \sigma _{\mu \nu} \, \mathsf{d} x ^{\mu} \otimes \mathsf{d} x ^{nu}\).
\end{defn}

\begin{defm}{Wedge Product (1-forms)}
	The \textbf{wedge product} is the fully antisymmetric tensor product.

	\(\mathsf{d} x ^{\mu} \wedge \mathsf{d} x ^{\nu} = - \mathsf{d} x ^{\nu} \wedge  \mathsf{d} x ^{\mu} := 2 \mathsf{d} x ^{[\mu} \otimes \mathsf{d} x ^{\nu ]} = \mathsf{d} x ^{\mu} \otimes \mathsf{d} x ^{\nu} - \mathsf{d} x ^{\nu} \otimes \mathsf{d} x ^{\mu}\).
\end{defm}

\begin{defm}{\(k\)-Form}
	A \textbf{\(k\)-form}, \(\mathsf{\sigma}\), is a fully antisymmetric \(k\)-times covariant (type \(\left( 0, k \right) \)) tensor.

	\(\mathsf{\sigma} = \sigma \indices{_{\alpha _{1} \dots \alpha _{k}}} \mathsf{d} x ^{\alpha _{1}} \otimes \cdots \otimes \mathsf{d} x ^{\alpha _{k}} = \frac{1}{k !} \sigma \indices{ _{\alpha _{1} \dots \alpha _{k}}} \mathsf{d} x ^{\alpha _{1}} \wedge \cdots \wedge \mathsf{d} x ^{\alpha _{k}}\).
\end{defm}

\begin{defm}{Exterior Product (Wedge Product)}
	If \(\mathsf{\sigma}\) is a \(k\)-form and \(\mathsf{\omega}\) is a \(p\)-form. The \textbf{exterior product} of \(\mathsf{\sigma}\) and \(\mathsf{\omega}\) is their fully antisymmetrized tensor product.

	\(\left( \mathsf{\sigma} \wedge \mathsf{\omega} \right) \indices{_{\alpha _{1} \dots \alpha _{k} \beta _{1} \dots \beta _{p}}} = \frac{\left( k + p \right) !}{k! p!} \sigma \indices{_{[\alpha _{1} \dots \alpha _{k}}} \omega \indices{_{\beta _{1} \dots \beta _{p}]}}\).
\end{defm}

\begin{defm}{Exterior Derivative}
	The \textbf{exterior derivative} \(\mathsf{d}\), is an operation that acts on a \(k\)-form and return a \(\left( k + 1 \right) \)-form.
	\begin{equation*}
		\begin{split}
			\mathsf{d} \mathsf{\sigma} &= \frac{1}{k !} \mathsf{d} \left( \sigma \indices{_{\alpha _{1} \dots \alpha _{k}}} \right) \wedge \mathsf{d} x ^{\alpha _{1}} \wedge \cdots \wedge \mathsf{d} x ^{\alpha _{k}} \\
			&= \frac{1}{k !} \partial _{\beta} \left( \sigma \indices{_{\alpha _{1} \dots \alpha _{k}}} \right) \mathsf{d} x ^{\beta} \wedge \mathsf{d} x ^{\alpha _{1}} \wedge \cdots \wedge \mathsf{d} x ^{\alpha _{k}} \\
			&= \left( k + 1 \right) \partial _{[ \beta} \sigma \indices{_{\alpha _{1} \dots \alpha _{k}]}} \mathsf{d} x ^{\beta} \otimes \mathsf{d} x ^{\alpha _{1}} \otimes \cdots \otimes \mathsf{d} x ^{\alpha _{k}}
		\end{split}
	\end{equation*}
\end{defm}

\begin{defm}{Exact}
	A \(k\)-form, \(\mathsf{\omega}\), is \textbf{exact} \(\iff \mathsf{\omega} = \mathsf{d} \mathsf{\sigma}\), where \(\mathsf{\sigma}\) is a \(\left( k - 1 \right) \)-form.
\end{defm}

\begin{defm}{Closed}
	A \(k\)-form, \(\mathsf{\omega}\), is \textbf{closed} \(\iff \mathsf{d} \mathsf{\omega} = 0\).
\end{defm}

\begin{thm}{Exact \(\implies\) Closed}
	\(\mathsf{\omega} = \mathsf{d} \mathsf{\sigma} \implies \mathsf{d} \mathsf{\omega} = 0\). However, the converse is not true in general.

	\(\mathsf{d} \mathsf{\omega} = 0 \implies \mathsf{\omega} = \mathsf{d} \mathsf{\sigma}\) locally.
\end{thm}

\begin{thm}{Poincar\'e Lemma}	
	If \(\mathsf{\omega}\) is defined in a contractible domain (open subset of a manifold), \(\mathsf{d} \mathsf{\omega} = 0 \iff \mathsf{\omega} = \mathsf{d} \mathsf{\sigma}\).
\end{thm}

\begin{noot}
	When solving exact ODEs, we are actually have \(\mathsf{d} F = 0\) and we check \(\partial _{y} \partial _{x} F = \partial _{x} \partial _{y} F\) which we are actually checking for closeness.
\end{noot}

\begin{remarkbox}
	If \(\mathsf{\omega}\) is exact then \(\mathsf{\omega} = \mathsf{d} f \implies \int _{\gamma} \mathsf{\omega} = \int _{\gamma} \mathsf{d} f = f(s _{2}) - f (s _{1})\). Also, \(\oint _{\gamma} \mathsf{\omega} = 0\).

	In polar coordinate, \(\mathsf{d} \mathsf{\theta}\) is actually not exact as \(\oint _{\gamma} \mathsf{d} \mathsf{\theta} = 2 \pi n \neq 0\). Also, \(\Theta = \mathsf{d} \mathsf{\theta}\) is not defined at the origin.
\end{remarkbox}

\subsection{Smooth Maps and Diffeomorphisms}
\subsubsection{Smooth Maps: Pullback and Pushforward}
\begin{defn}{Smooth Map}
	A map \(\varphi : \mathcal{M} \to \mathcal{M}'\) is a \textbf{smooth map} if and only if given  the atlases\(\left\{ \left( \mathcal{U} _{\alpha}, \phi _{\alpha} \right)  \right\} \) and \(\left\{ \mathcal{U}' _{\alpha}, \phi ' _{\alpha} \right\} \) of \(\mathcal{M}\) and \(\mathcal{M}'\), the functions \(\phi ' _{\alpha} \circ \varphi \circ \phi _{\alpha}: \R ^{n} \to \R ^{n'}\) is smooth.
\end{defn}

\begin{defm}{Pullback (Function)}
	The map \(\varphi\) induces a map \(\varphi ^{\star}: \mathcal{F} _{\mathcal{M}} \to \mathcal{F} _{\mathcal{M}'}\) which we call the \textbf{pullback} between the spaces of functions \(\mathcal{F} _{\mathcal{M}}\) and \(\mathcal{F} _{\mathcal{M'}}\) according to the following rule:

	Given a function \(f' : \mathcal{M} ' \to \R\), its \textbf{pull-back} \(\varphi ^{\star} f': \mathcal{M} \to \R\) is such that \(\varphi ^{\star} f' (p) := f' \circ \varphi (p) = f' (p)\), with \(p \in \mathcal{M}\).
\end{defm}

\begin{defm}{Pushforward (Vector)}
	The map \(\varphi\) induces a map \(\varphi _{\star}: T _{p} \mathcal{M} \to T _{\varphi (p)} \mathcal{M'}\) called the \textbf{pushforward} between the tangent spaces \(T _{p} \mathcal{M}\) and \(T _{p} \mathcal{M'}\) according to the following rule:
	
	Given a vector \(\mathsf{v} \in T _{p} \mathcal{M}\) then \(\varphi _{\star} \mathsf{v} \in T _{p} \mathcal{M'}\) such that its action on a function \(f' \in \mathcal{F} _{\mathcal{M}'}\) is \(\left( \varphi _{\star} \mathsf{v} \right) | _{\phi (p) = p'} (f') := \mathsf{v} | _{p} \left( \varphi ^{\star} f' \right) \).
	\begin{remarkbox}
		In coordinate bases \(\left( x ^{1}, \dots, x ^{m} \right) \) and \(\left( y ^{1}, \dots, y ^{m'} \right) \):

		\(\left( [\varphi _{\star}] \mathsf{v} \right) | _{\varphi (p)} (f') = v ^{i} | _{p} \frac{\partial f'}{\partial y ^{j}} \frac{\partial y ^{j}}{\partial x ^{i}} | _{p} \implies \left( [\varphi _{\star}] \mathsf{v} \right) | _{\varphi (p)} = v ^{i} | _{p} \frac{\partial y ^{j}}{\partial x ^{i}} | _{p} \frac{\partial}{\partial y ^{j}} | _{\partial (p)}\).
	\end{remarkbox}
	\begin{remarkbox}
		If the map \(\varphi\) is not injective, than the pushforward of a vector field does not define a vector field.
	\end{remarkbox}
\end{defm}

\begin{defm}{Pullback (one-form)}
	Given a 1-form \(\mathsf{\omega} \in T _{p} ^{\star} \mathcal{M'}\) then \(\varphi ^{\star}: T _{\varphi (p)} ^{\star} \mathcal{M}' \to T _{p} ^{\star} \mathcal{M}\) called the \textbf{pullback} between the cotangent spaces \(T _{\varphi (p)} ^{\star} \mathcal{M}'\) and \(T _{p} ^{\star} \mathcal{M}\) according to the following rule:
	Given a 1-form \(\mathsf{\omega}' \in T _{\varphi (p)} ^{\star} \mathcal{M'}\), then \(\varphi ^{\star} \mathsf{\omega}' \in T _{p} ^{\star} \mathcal{M}\) whose action on vectors in \(T _{p} \mathcal{M}\) is \(\left<\varphi ^{\star} \mathsf{\omega}', \mathsf{v} \right> | _{p} := \left< \mathsf{\omega}', \varphi _{\star} \mathsf{v} \right> | _{\varphi (p)}\).
	\begin{remarkbox}
		\(\mathsf{d} \left( \varphi ^{\star} \mathsf{\omega} \right) = \varphi ^{\star} \left( \mathsf{d} \mathsf{\omega} \right)  \).
	\end{remarkbox}
	\begin{remarkbox}
		\(\varphi ^{\star} \left( \mathsf{\alpha} \wedge \mathsf{\beta} \right) = \left( \varphi ^{\star} \mathsf{\alpha} \right)\wedge \left( \varphi ^{\star} \mathsf{\beta} \right) \).
	\end{remarkbox}
	\begin{remarkbox}
		In coordinate basis, \(\varphi ^{\star} \mathsf{\omega} = \left( \omega _{i} \circ \varphi \right)  \mathsf{d} \left( x ^{i} \circ \varphi \right)\).
	\end{remarkbox}
\end{defm}

\begin{defm}{Pullback (k-form)}
	\(\varphi ^{\star} \mathsf{\omega} = \varphi ^{\star} \left(\frac{1}{k !} \omega \indices{_{\alpha _{1} \dots \alpha _{k}}} \bigwedge _{i = 1} ^{k} \mathsf{d}x ^{\alpha _{i}} \right)\).
\end{defm}

\subsubsection{Diffeomorphisms}
\begin{defm}{Diffeomorphism}
	A map \(\varphi: \mathcal{M} \to \mathcal{M}'\) is a \textbf{diffeomorphism} if and only if \(\varphi\) and \(\varphi ^{-1}\) are both smooth bijections.
	\begin{remarkbox}
		For a diffeomorphism \(\varphi: \mathcal{M} \to \mathcal{M}\), pullback and pushfoward can be defined for all objects on the manifold as \(\varphi _{\star} = \left( \varphi ^{-1} \right) ^{\star}\).
		\begin{remarkbox}
			\(\left[ \varphi _{\star} \right] | _{p} ^{-1} = \left[ \varphi _{\star} ^{-1} \right] | _{\varphi(p)}  \).

			\(\varphi _{\star} f(p) = \left( \varphi ^{-1} \right) ^{\star} f(p) \) and \(\left( \varphi ^{\star} \mathsf{v} \right) | _{p} (f) = \mathsf{v} _{\varphi ^{-1} (p)} \varphi _{\star} f\).
		\end{remarkbox}
	\end{remarkbox}
\end{defm}

\begin{defn}{Local Flow}
	Consider a diffeomorphism \(\varphi: \mathcal{M} \to \mathcal{M}\) and let \(\gamma _{p}(s)\) be the parametrized curve in \(\mathcal{M}\) such that \(\gamma _{p} (0) = p\) and its tangent vector at every points \(\gamma _{p} (s)\) is the vector field \(\mathsf{k}\). \(\mathsf{k}\) is the generator of a set of local diffeomorphisms of the form \(\varphi _{s}: \mathcal{U} \to \varphi _{s} (\mathcal{U})\) such that \(\varphi _{s} (p) = \gamma _{p} (s)\). This set of local diffeomorphisms is called the \textbf{local flow} of \(\mathsf{k}\).
	\begin{noot}
		\begin{enumerate}[label=\arabic*)]
			\item \(\varphi _{-s} \circ \varphi _{s} (p) = p \implies \varphi _{s} ^{-1} = \varphi _{-s}\)
			\item \(\varphi _{s} (p) = \gamma (s)\)
			\item \(\varphi _{s + t} (p) = \gamma (s + t)\)
			\item \(\phi _{s} \left( \phi _{t} (p) \right) = \varphi _{s + t} (p)\)
		\end{enumerate}
	\end{noot}
	Hence, local diffeomorphisms forms a group (Lie group). And the local flow of the vector field \(\mathsf{k}\) is complete if the associated curve \(\gamma _{p} (s)\) can be extended for all values \(s \in \R\).
\end{defn}

\subsubsection{Lie Derivative}
\textbf{Motivation:} Let \(\varphi: \mathcal{M} \to \mathcal{M}\) be a diffeomorphism, \(\varphi\) establishes an isomorphism \(\varphi ^{\star}: \left( T _{\varphi (p)} \right) ^{r} _{s} \mathcal{M} \to \left( T _{p} \right) ^{r} _{s} \mathcal{M} \) for \(p \in \mathcal{M}\). The tensor \(\left[ \varphi ^{\star} \left( T _{\varphi (p)} \right)  \right] |_{p} - T |_{p} \) provides information about the difference between the value of a tensor field at \(\varphi (p)\) and at \(p\).

\begin{defm}{Lie Derivative}
	Consider a diffeomorphism \(\varphi : \mathcal{M} \to \mathcal{M}\) and its pullback \(\varphi ^{\star}: \left( T _{\varphi (p)} \right) ^{r} _{s} \mathcal{M} \to \left( T _{p} \right) ^{r} _{s} \mathcal{M}  \). Consider \(\varphi \equiv \varphi _{t}\) be the local flow generated by a vector field \(\mathsf{v}\).
	The \textbf{Lie derivative} of a tensor \(\mathsf{T}\) along the (curve generated by) vector field \(\mathsf{v}\) at the point \(p \in \mathcal{M}\) is \(\left( \mathcal{L} _{\mathsf{v}} \mathsf{T} \right) | _{p} :=  \lim _{t \to 0} \frac{\left[ \varphi _{t} ^{\star} \left( \mathsf{T} _{\varphi _{t} (p)} \right)  \right] _{p} - T _{p} }{t}\).
	\begin{noot}
		\(t\) is a parameter along a curve generated by \(\mathsf{v}\).

		Intuitively, the Lie derivative is to:
		\begin{enumerate}[label=\arabic*)]
			\item Take the tensor at t steps away from point \(p \in \mathcal{M}\) (along the curve generated by \(\mathsf{v}\)).
			\item Move the tensor to point \(p\) by using the pullback.
			\item Compute the difference between the two tensors at \(p\).
			\item Divide the difference of the two tensors by \(t\) then take the limit \(t \to 0\).
		\end{enumerate}
		Acknowledgement: This intuition is from \href{https://file.notion.so/f/f/60cb0192-2769-448c-b567-2b1869af954b/9094c6a5-938a-43d1-bbbc-35bbf9c96ffd/%E6%82%A6%E6%82%A6%E7%88%B1%E7%89%A9%E7%90%86%E5%B9%BF%E4%B9%89%E7%9B%B8%E5%AF%B9%E8%AE%BA%E7%AC%94%E8%AE%B02.7%E7%89%88%E6%9C%AC_GR_Notes_Differential_Geometry_and_General_Relativity_for_Beginners_by_Jiayue_Yang_(17).pdf?table=block&id=6c51f98a-7d8e-4071-a36c-1e23b89a8ed6&spaceId=60cb0192-2769-448c-b567-2b1869af954b&expirationTimestamp=1740916800000&signature=byzCKVB4bTNGJ-TXycNArj8XU5ZTh_T-w2VL_lcTssA&downloadName=%E6%82%A6%E6%82%A6%E7%88%B1%E7%89%A9%E7%90%86%E5%B9%BF%E4%B9%89%E7%9B%B8%E5%AF%B9%E8%AE%BA%E7%AC%94%E8%AE%B02.7%E7%89%88%E6%9C%AC+GR_Notes_Differential_Geometry_and_General_Relativity_for_Beginners_by_Jiayue_Yang+%2817%29.pdf}{here}.
		One can also think of Lie Derivative as something that measures the variation of a tensor field when moving 
	\end{noot}
	\begin{remarkbox}
		Lie Derivative have the properties:
		\begin{enumerate}[label=\arabic*)]
			\item Linearity.
			\item Preserves the tensor type, symmetries, and contractions because \(\varphi ^{\star}\) does.
			\item Satisfies Leibniz rule under tensor product and contraction: \(\mathcal{L} _{\mathsf{v}}\left( \mathsf{T} \otimes \mathsf{S} \right) = \left( \mathcal{L} _{\mathsf{v}} \mathsf{T} \right) \otimes \mathsf{S} + \mathsf{T} \otimes \left( \mathcal{L} _{\mathsf{v}} \mathsf{S} \right) \).
			\item Cartan identity: The Lie derivative of a k-form field along the vector field \(\mathsf{v}\) follows \(\mathcal{L} _{\mathsf{v}} = i _{\mathsf{v}} \mathsf{d} + \mathsf{d} i _{v}\) (took from JiayuePhysics).
		\end{enumerate}
	\end{remarkbox}
\end{defm}

\begin{defm}{Lie Derivative of Scalar Field}
	For a scalar field \(f\), \(\mathcal{L} _{\mathsf{v}} f = \mathsf{v} (f)\), and in a coordinate basis, \(\mathcal{L} _{\mathsf{v}} f = v ^{\nu} \partial _{\nu} f\).
\end{defm}

\begin{defm}{Lie Derivative of Vector Field}
	For a vector field \(\mathsf{w}\), \(\mathcal{L} _{\mathsf{v}} \mathsf{w} = \left[ \mathsf{v}, \mathsf{w} \right] \), and in a coordinate basis, \(\mathcal{L} _{\mathsf{v}} \mathsf{w} = \left( v ^{\nu} \partial _{\nu} w ^{\mu} - w ^{\nu} \partial _{\nu} v ^{\mu} \right) \partial _{\mu}\).
\end{defm}

\begin{defm}{Lie Derivative of 1-Form Field}
	For a 1-form field \(\omega\) in a coordinate basis, \(\mathcal{L} _{\mathsf{v}} \mathsf{\omega} = \left( v ^{\nu} \left( \partial _{\nu} \omega _{\mu} \right) + \omega _{\nu} \left( \partial _{\mu} v ^{\nu} \right)   \right) \mathsf{d} x ^{\mu}\).
\end{defm}

\begin{ex}{More Lie Derivatives}
	\(\omega = \omega _{\mu \nu} \mathsf{d} x ^{\mu} \otimes \mathsf{d} x ^{\nu} \implies \mathcal{L} _{\mathsf{v}} \mathsf{\omega} = \left( v ^{\alpha} \left( \partial _{\alpha} \omega _{\mu \nu}\right) + \omega _{\alpha \nu} \left( \partial _{\mu} v ^{\alpha} \right) + \omega _{\mu \alpha} \left( \partial _{v} v ^{\alpha} \right)  \right) \mathsf{d} x ^{\mu} \otimes \mathsf{d} x ^{\nu}\).

	\(\mathsf{T} = T \indices{_{\mu} ^{\nu}} \mathsf{d} x ^{\mu} \otimes \partial _{\nu} \implies \mathcal{L} _{\mathsf{v}} \mathsf{T} = \left( v ^{\alpha} \left( \partial _{\alpha} T \indices{_{\mu} ^{\nu}} \right) + T \indices{_{\alpha} ^{\nu}} \left( \partial _{\mu} v ^{\alpha} \right) - T \indices{_{\mu} ^{\alpha}} \left( \partial _{\alpha} v ^{\nu} \right)  \right) \mathsf{d} x ^{\mu} \otimes \partial _{\nu}\).

	\(\mathsf{T} = T \indices{_{\alpha \beta} ^{\mu \nu}} \mathsf{d} x ^{\alpha} \otimes \mathsf{d} x ^{\beta} \otimes \partial _{\mu} \otimes \partial _{\nu}\).

	\(\implies \mathcal{L} _{\mathsf{v}} \mathsf{T} = \left( v ^{\gamma} \partial _{\gamma} T \indices{_{\alpha \beta} ^{\mu \nu}} + T \indices{_{\gamma \beta} ^{\mu \nu}} \partial _{\alpha} v ^{\gamma} + T \indices{_{\alpha \gamma} ^{\mu \nu}} \partial _{\beta} v ^{\gamma} - T \indices{_{\alpha \beta} ^{\gamma \nu}} \partial _{\gamma} v ^{\mu} - T \indices{_{\alpha \beta} ^{\mu \gamma}} \partial _{\gamma} v ^{\nu} \right) \mathsf{d} x ^{\alpha} \otimes \mathsf{d} x ^{\beta} \otimes \partial _{\mu} \otimes \partial _{\nu} \).
\end{ex}

\subsection{Pseudo-Riemannian Geometry}
\subsubsection{Affine Connection and Covariant Derivative}
\textbf{Motivation:} While Lie derivative is a derivative, we can't use it as a generalised gradient, as it doesn't change the type of the tensor. So we need to explore other derivatives. We take the derivative of a function \(f\) as \(\frac{\partial f}{\partial x ^{\mu}} = \partial _{\mu} f\). And for a vector \(\mathsf{v} = v ^{\mu} \mathsf{\Upsilon_{\mu}}\), \(\frac{\partial \mathsf{v}}{\partial x ^{\mu}}\) is a bad notation for the derivative of the vector because both the components and basis of \(\mathsf{v}\) can change. What we want to do is to write \(\frac{\partial \mathsf{\Upsilon}}{\partial x ^{\mu}}\) in terms of the bases (\(\frac{\partial \Upsilon}{\partial x ^{\mu}} = \Gamma \indices{^{\rho} _{\nu \mu}} \Upsilon _{\rho}\) with \(\Gamma \indices{^{\rho} _{\nu \mu}}\) describing how basis changes at different point).

\begin{defm}{Affine Connection, Covariant Derivative}
	An \textbf{affine connection} \(\nabla: \left( T _{p} \mathcal{M} \right) ^{r} _{s} \to \left( T _{p} \mathcal{M} \right) ^{r} _{s + 1}\) is a rule through which we assign to each tensor \(\mathsf{T}\) of type \(\left( r, s \right) \) and components \(T \indices{^{b c \dots} _{d e \dots}}\) another tensor field \(\nabla \mathsf{T}\) of type \(\left( r, s + 1 \right) \) and components \(\left( \nabla \mathsf{T} \right) \indices{^{b c \dots} _{d e \dots a}} \equiv \nabla _{a} T \indices{^{b c \dots} _{d e \dots}} \equiv T \indices{^{b c \dots} _{d e \dots; a}}\). We call \(\nabla \mathsf{T}\) the \textbf{covariant derivative} of \(\mathsf{T}\).
	\begin{remarkbox}
		Affine connection follows the properties:
		\begin{enumerate}[label=\arabic*)]
			\item Linearity.
			\item Leibniz rule.
			\item Commutes with contraction.
			\item Over functions \(f\), it is the differential \(\nabla f = \mathsf{d} f\).
		\end{enumerate}
	\end{remarkbox}
	\begin{noot}
		There are infinitely many affine connections and infinite ways to build covariant derivative.
	\end{noot}
\end{defm}

\begin{defm}{Directional Covariant Derivative}
	We define the \textbf{directional covariant derivative} of \(\mathsf{T}\) in the direction of vector \(\mathsf{v}\) as the tensor \(\nabla _{\mathsf{v}} \mathsf{T}\) of type \(\left( r, s \right) \) of components \(\left( \nabla _{\mathsf{v}} \mathsf{T} \right) \indices{^{b c \dots} _{de \dots}} = v ^{a} \nabla _{a} T \indices{^{b c \dots} _{d e \dots}}\). \(\left( \nabla _{\mathsf{v}} \mathsf{T} \right) \indices{^{bc \dots} _{de \dots}} = \left< \nabla \mathsf{T}, \mathsf{v} \right> \).
\end{defm}

\begin{defm}{Coefficient of the Connection}
	The \textbf{coefficient of the connection} in an arbitary basis \(\left\{ \mathsf{\Upsilon} _{a} \right\} \) is
	\[
		\Gamma \indices{^{a} _{bc}}:= \left( \nabla \mathsf{\Upsilon} _{b} \right) \indices{^{a} _{c}} = \left< \mathsf{\Upsilon} ^{a}, \nabla _{\mathsf{\Upsilon} _{c}} \mathsf{\Upsilon} _{b} \right> \implies \nabla _{c} \mathsf{\Upsilon} _{b} := \nabla _{\mathsf{\Upsilon} _{c}} \mathsf{\Upsilon} _{b} := \Gamma \indices{^{a} _{bc}} \mathsf{\Upsilon} _{a}.
	\]
	\(\Gamma \indices{^{a} _{bc}}\) is the \(a\)-th components of the covariant derivative of the \(b\)-th basis vector in the direction of the \(c\)-th basis vector.
	\begin{remarkbox}
		The notation for covariant derivative \(\nabla _{\mu} v ^{\nu} = \left( \nabla \mathsf{v} \right) \indices{_{\mu} ^{\nu}} \). Also...
	\end{remarkbox}
	\begin{noot}
		Another convention is \(\Gamma \indices{^{a} _{bc}} := \left( \nabla \mathsf{\Upsilon} _{c} \right) \indices{^{a} _{b}} = \left< \mathsf{\Upsilon} ^{a}, \nabla _{\mathsf{\Upsilon} _{b}} \mathsf{\Upsilon} _{c} \right> \).
	\end{noot}
\end{defm}
\(\nabla _{a} v ^{b} = \Upsilon _{a} (v ^{b}) + v ^{c} \Gamma \indices{^{b} _{c a}}\)

\(\nabla _{\mu} v ^{\nu} = \partial _{\mu} v ^{\nu} + v ^{\sigma} \Gamma \indices{^{\nu} _{\sigma \mu}}\)

\(\nabla _{a} \omega _{b} = \Upsilon _{a} (\omega _{b}) - \omega _{c} \Gamma \indices{^{c} _{ba}}\)

\(\nabla _{\mu} \omega _{\nu} = \partial _{\mu} \omega _{nu} - \omega _{\sigma} \Gamma \indices{^{\sigma} _{\nu \mu}}\)

Note:
\(\frac{\partial \mathsf{v}}{\partial x ^{\mu}} = \left( \frac{\partial v ^{\nu}}{\partial x ^{\mu}} + v ^{\sigma} \Gamma \indices{^{\nu} _{ \sigma \mu}} \right) \Upsilon _{\nu}\).
\end{document}
Note: mass charge doesn't affect acceleration. (GR)
\end{document}
